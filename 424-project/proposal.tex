\documentclass[11pt]{article}
\usepackage[pdftex]{graphicx}
\usepackage{amsmath}
\usepackage{amssymb}
\usepackage{fancyhdr}
\usepackage{bbold}
\usepackage{hyperref}
\usepackage[hmargin=2cm,vmargin=3cm]{geometry}
\pagestyle{fancy}
\fancyhead{}
\fancyfoot{}
\fancyhead[LO]{Colleen Carroll, Nicolas Crowell,\\ 
	       Elan Kugelmass, and Andrew Morrison}
\fancyhead[CO]{4/3/14}
\fancyhead[RO]{\thepage}
\renewcommand{\headrulewidth}{1pt}
\renewcommand{\footrulewidth}{1pt}
\renewcommand{\baselinestretch}{1}

\DeclareMathOperator{\Like}{\mathcal L}
\DeclareMathOperator{\Ind}{\mathbb{1}}
\DeclareMathOperator{\Exp}{\mathop{\mathbb{E}}}
\DeclareMathOperator{\N}{\mathcal{N}}

\begin{document}
\setcounter{section}{0}
\setcounter{subsection}{0}
\setcounter{subsubsection}{0}
\ \ \\
COS424 \\
Final project proposal
\\

\subsection*{What is the problem? Why work on it?}
It is thought that mammals maintain neurological representations of their 
reference frame and context in the hippocampus. This map is maintained in a set 
of cells called `place cells.' We can observe activity in this cell cluster 
using electrodes. Rats are observed while performing a task. At certain points, 
the reference
frame, i.e. the task assigned to a rat,
 is abruptly changed. The `place cell' theory predicts that the firing pattern
of place cells should change upon remapping, in order to reflect the new
reference frame the rat has established. The precise relationship between
shifts of context and place cell activity has not yet been characterized.
We will develop a classifier to relate nerve activity patterns to specific
rat behaviors and exogenous environmetal changes.

The complication of this problem is that we do not have a particularly
good understanding of why certain place cells fire, and there is a substantial
amount of extraneous neurological noise in the observations. Individual place
cells may represent different locations after context remappings, or more than
one location within a context. At a given moment, it is possible that no
cells are firing, perhaps because there is no place cell corresponding to the
current location of the rat. Noise from the data collection process, including
neurological noise from adjacent cell firings, make this a ripe opportunity
for machine learning.

\subsection*{What data are available?}
Data collected from eight CA1 tetrodes have been recorded from a rat performing 
two
 distinct behaviors in virtual reality. Electrodes from a single tetrode are 
spaced far 
enough apart so that typical electric waves register different amplitudes, but 
close
 enough together so that all electrodes measure a wave if one does. This 
assumption 
allows spike data to be sorted according to which neuron generated the spike. 
Training
 data for supervised learning can be obtained from neural recordings taken when 
a rat is
 unambiguously attending to one task or the other.


\subsection*{What methods have others tried?}
Within neuroscience, the question of the function of place cells ha been 
explored through 
various experiments to determine what external stimuli will cause them to fire. 
Earlier studies used a dot product similarity threshold-based statistic [1], as 
well as
applying a very general Bayesian framework and kernel density estimation 
technique [2]
to try to use neuronal spike data to do spatial localization. However, no 
research has yet been done to 
determine the effectiveness of applying machine learning classification 
techniques to 
determining the rat's neuroloical state during the transition period between 
tasks. 

[1]: Jezek, Henrikson, Treves, Moser, Moser, 2011
[2]: Kloosterman, 2013

\subsection*{What methods do we plan to explore or develop?}
We hypothesize that this problem is relatively well-modeled by a hidden
Markov model, in which the latent state is a categorical random variable
where the values of the variable correspond to the current mental context
of the rat subject. In this experiment, the context is believed to correspond 
to
the task which the rat currently believes it is completing. We also plan to 
explore 
conditional random fields, which may offer another way to use the relational 
information 
among the observations, as well as much of the information belonging to each 
observation.


\subsection*{What does success mean? How will we evaluate it?}
Running our classifier on the rat's neurological state would allow us to predict 
the rat's behavior at 
different timepoints in the experiment, for which we have labelled data to 
compare against.
Temporally close predictions would mean a successful classifier.

\subsection*{What challenges do we expect to face?}
Determining the rat's nuerological state depends on several factors. This is 
not a simple matter of 
determining which place cells are firing while the rat is completing a certain 
task. The place cells do not
necessarily fire constantly while the rat is in a certain context. It is also 
uncertain whether place cells 
actually correspond directly to a location in space or some other feature or 
set of features of the 
rat's environment. These among other uncertainties about why a place cell fires 
could mean that 
we have difficulty determining the features to consider in our model and which 
distributions would 
best represent them. However, getting a better understanding of the importance 
of certain features 
can bring us closer to our goal of understanding the role of place cells.

\end{document}
